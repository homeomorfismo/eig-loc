% Pgfplots config
\pgfplotscreateplotcyclelist{cblist}{
    {T-Q-B1, thick, mark=o},
    {T-Q-B2, thick, mark=+},
    {T-Q-B3, thick, mark=otimes},
    {T-Q-B4, thick, mark=triangle*},
    {T-Q-B5, thick, mark=diamond*},
    {T-Q-B6, thick, mark=square},
    {T-Q-HC0,mark=x}
}

% (Real part of) eigenvalues of the Laplacian on the unit square with Dirichlet boundary conditions
% close to the interval \([50\pi^2 - 10, 50\pi^2 + 10]\).
\begin{figure}[H]
    \centering
    \begin{tikzpicture}
        \begin{semilogxaxis}[
            xlabel={Number of degrees of freedom},
            ylabel={Eigenvalues},
            legend pos=south west,
            grid=major,
            cycle list name=cblist,
            xmin=150,
            xmax=120000,
            clip=false,
            forget plot style={black!75, dotted},
            legend entries={
                {\(\lambda_{i,n}\)},
                {\(\lambda_{i,N}\)},
            },
        ]
        % Add data
        \addplot+ [only marks] table[x=Ndofs, y=Evals, col sep=comma] {eigenvalues_landscape.csv};

        % Add horizontal lines
        \addplot[forget plot, mark=none, samples=20] coordinates { (159, 485.7222447031804)  (120000, 485.7222447031804)};
        \addplot[forget plot, mark=none, samples=20] coordinates { (159, 490.1641386920197)  (120000, 490.1641386920197)};
        \addplot[forget plot, mark=none, samples=20] coordinates { (159, 493.48390547991005) (120000, 493.48390547991005)};
        \addplot[forget plot, mark=none, samples=20] coordinates { (159, 493.48401556963995) (120000, 493.48401556963995)};
        \addplot[forget plot, mark=none, samples=20] coordinates { (159, 493.48452662344147) (120000, 493.48452662344147)};
        \addplot[forget plot, mark=none, samples=20] coordinates { (159, 499.24505256207357) (120000, 499.24505256207357)};
        \addplot[forget plot, mark=none, samples=20] coordinates { (159, 502.3049699107303)  (120000, 502.3049699107303)};

        % Add markers
        \node [pin=right:{\(\lambda_{0}\)}] at (axis cs: 100000, 485.7222447031804) {};
        \node [pin=right:{\(\lambda_{1}\)}] at (axis cs: 100000, 490.1641386920197) {};
        \node [pin=right:{\(\lambda_{2, 3, 4}\)}] at (axis cs: 100000, 493.48390547991005) {};
        \node [pin=right:{\(\lambda_{5}\)}] at (axis cs: 100000, 499.24505256207357) {};
        \node [pin=right:{\(\lambda_{6}\)}] at (axis cs: 100000, 502.3049699107303) {};

        \end{semilogxaxis}
    \end{tikzpicture}
    \caption{
        Eigenvalues inside the interval \([50\pi^2 - 10, 50\pi^2 + 10]\).
        \(\lambda_{i,n}\) denotes the \(i\)-th eigenvalue at \(n\)-th iteration, and \(\lambda_{i,N}\) denotes the \(i\)-th eigenvalue at the final iteration.
        Here, \(\lambda_{0,N} = 485.7222447031804\), \(\lambda_{1,N} = 490.1641386920197\), \(\lambda_{2,N} \approx \lambda_{3,N} \approx \lambda_{4,N} \approx 50\pi^2\), \(\lambda_{5,N} = 499.24505256207357\), and \(\lambda_{6,N} = 502.3049699107303\).
    }
    \label{fig:eigenvalues}
\end{figure}


% Evolution of the error estimator \(\eta := \eta(u)\) as a function of the number of degrees of freedom.
\begin{figure}[H]
    \centering
    \begin{tikzpicture}
        \begin{loglogaxis}[
            xlabel={Number of degrees of freedom},
            ylabel={Error estimator},
            legend entries={\(\eta\)},
            legend pos=south west,
            cycle list name=cblist,
            grid=major,
            forget plot style={black!75, dotted},
        ]
        \addplot+ table[x=Ndofs, y={Eta}, col sep=comma] {error_estimator_landscape.csv};
        \addplot [forget plot] table [
            col sep=comma,
            x=Ndofs,
            y={create col/linear regression={
                y=Eta
                % variance list={1000,800,600,500,400,200,100}
                }
            }] {error_estimator_landscape.csv}
            coordinate [pos=0.8] (A)
            coordinate [pos=0.9] (B);
        \xdef\slope{\pgfplotstableregressiona}
        \draw (A) -| (B) node[pos=0.75, anchor=west] {\(\pgfmathprintnumber{\slope}\)};
        \end{loglogaxis}
    \end{tikzpicture}
    \caption{Error estimator \(\eta := \eta(u)\) as a function of the number of degrees of freedom.}
    \label{fig:error}
\end{figure}

% Monitoring the behavior of the error estimator \(\eta_i := \eta(\psi_i, \lambda_i)\) as a function of the number of degrees of freedom.
\begin{figure}[H]
    \centering
    \begin{tikzpicture}
        \begin{loglogaxis}[
            xlabel={Number of degrees of freedom},
            ylabel={Error estimator},
            legend entries={{\(\eta_0\)},
                            {\(\eta_1\)},
                            {\(\eta_2\)},
                            {\(\eta_3\)},
                            {\(\eta_4\)},
                            {\(\eta_5\)},
                            {\(\eta_6\)}},
            legend pos=south west,
            cycle list name=cblist,
            grid=major,
            forget plot style={black!75, dotted},
        ]
        % \addplot table[x=Ndofs, y={Eta}, col sep=comma] {error_estimator.csv};
        \addplot+ table[x=Ndofs, y=Eta Eigenvalue 0, col sep=comma] {error_estimator_landscape.csv};
        \addplot+ table[x=Ndofs, y=Eta Eigenvalue 1, col sep=comma] {error_estimator_landscape.csv};
        \addplot+ table[x=Ndofs, y=Eta Eigenvalue 2, col sep=comma] {error_estimator_landscape.csv};
        \addplot+ table[x=Ndofs, y=Eta Eigenvalue 3, col sep=comma] {error_estimator_landscape.csv};
        \addplot+ table[x=Ndofs, y=Eta Eigenvalue 4, col sep=comma] {error_estimator_landscape.csv};
        \addplot+ table[x=Ndofs, y=Eta Eigenvalue 5, col sep=comma] {error_estimator_landscape.csv};
        \addplot+ table[x=Ndofs, y=Eta Eigenvalue 6, col sep=comma] {error_estimator_landscape.csv};
        \end{loglogaxis}
    \end{tikzpicture}
    \caption{Error estimator \(\eta_i := \eta(\psi_i, \lambda_i)\) as a function of the number of degrees of freedom.}
    \label{fig:error}
\end{figure}


% Visualizing the error of the eigenvalues, after landscape-driven refinement.
\begin{figure}[H]
    \centering
    \begin{tikzpicture}
        \begin{loglogaxis}[
            forget plot style={black!75, dotted},
            xlabel={Number of degrees of freedom},
            ylabel={\(\left\lvert \lambda_{i,n} - \lambda_{i,N} \right\rvert\)},
            cycle list name=cblist,
            legend pos=south west,
            grid=major,
            % xmin=150, xmax=257000,
            legend entries={\(\operatorname{err}_0\),
                            \(\operatorname{err}_1\),
                            \(\operatorname{err}_2\),
                            \(\operatorname{err}_3\),
                            \(\operatorname{err}_4\),
                            \(\operatorname{err}_5\),
                            \(\operatorname{err}_6\)},
        ]
        \addplot+ table[x=Ndofs, y=Error Eigenvalue 0, col sep=comma] {error_eigenvalues_landscape.csv};
        \addplot+ table[x=Ndofs, y=Error Eigenvalue 1, col sep=comma] {error_eigenvalues_landscape.csv};
        \addplot+ table[x=Ndofs, y=Error Eigenvalue 2, col sep=comma] {error_eigenvalues_landscape.csv};
        \addplot+ table[x=Ndofs, y=Error Eigenvalue 3, col sep=comma] {error_eigenvalues_landscape.csv};
        \addplot+ table[x=Ndofs, y=Error Eigenvalue 4, col sep=comma] {error_eigenvalues_landscape.csv};
        \addplot+ table[x=Ndofs, y=Error Eigenvalue 5, col sep=comma] {error_eigenvalues_landscape.csv};
        \addplot+ table[x=Ndofs, y=Error Eigenvalue 6, col sep=comma] {error_eigenvalues_landscape.csv};


        \addplot [forget plot] table[
            x=Ndofs,
            y={create col/linear regression={y=Error
            Eigenvalue 0}},
            col sep=comma] {error_eigenvalues_landscape.csv}
            coordinate [pos=0.8] (A)
            coordinate [pos=0.9] (B);
        \xdef\slope{\pgfplotstableregressiona}
        \draw (A) -| (B) node[pos=0.75, anchor=west] {\(\pgfmathprintnumber{\slope}\)};
        \end{loglogaxis}
    \end{tikzpicture}
    \caption{Error \(\operatorname{err}_i := \left\lvert \lambda_{i,n} - \lambda_{i,N} \right\rvert\) as a function of the number of degrees of freedom.}
    \label{fig:error_alt}
\end{figure}


% Table of convergence ratios, under landscape-driven refinement.
% TODO: Fix scientific notation!
% \begin{table}[h!]
%     \begin{center}
%         \label{tab:convergence}
%         \pgfplotstabletypeset[
%             col sep=comma,
%             columns={Ndofs, Ratio 0, Ratio 1, Ratio 2, Ratio 3, Ratio 4, Ratio 5, Ratio 6},
%             columns/Ndofs/.style={
%                 column name={\#\textsc{DoFs}},
%                 int detect,
%                 column type={c}
%             },
%             columns/Ratio 0/.style={
%                 column name={\(\operatorname{ord}_0\)},
%                 precision=2,
%                 fixed zerofill,
%                 column type={c}
%             },
%             columns/Ratio 1/.style={
%                 column name={\(\operatorname{ord}_1\)},
%                 precision=2,
%                 fixed zerofill,
%                 column type={c}
%             },
%             columns/Ratio 2/.style={
%                 column name={\(\operatorname{ord}_2\)},
%                 precision=2,
%                 fixed zerofill,
%                 column type={c}
%             },
%             columns/Ratio 3/.style={
%                 column name={\(\operatorname{ord}_3\)},
%                 precision=2,
%                 fixed zerofill,
%                 column type={c}
%             },
%             columns/Ratio 4/.style={
%                 column name={\(\operatorname{ord}_4\)},
%                 precision=2,
%                 fixed zerofill,
%                 column type={c}
%             },
%             columns/Ratio 5/.style={
%                 column name={\(\operatorname{ord}_5\)},
%                 precision=2,
%                 fixed zerofill,
%                 column type={c}
%             },
%             columns/Ratio 6/.style={
%                 column name={\(\operatorname{ord}_6\)},
%                 precision=2,
%                 fixed zerofill,
%                 column type={c}
%             },
%             ]{ratio_ev_landscape.csv}
%         \caption{
%             Order of convergence of the eigenvalues.
%             Here, \(\operatorname{err}_i \approx C_i N(h)^{-\operatorname{ord}_i}\).
%         }
%     \end{center}
% \end{table}

% Table of convergence (decay) ratio for the landscape error estimator.
\begin{table}
    \begin{center}
        \label{tab:error}
        \pgfplotstabletypeset[
            col sep=comma,
            columns={Ndofs, Ratio},
            columns/Ndofs/.style={
                column name={\#\textsc{DoFs}},
                int detect,
                column type={c}
            },
            columns/Ratio/.style={
                column name={\(\operatorname{ord}\)},
                precision=2,
                fixed zerofill,
                column type={c}
            },
            ]{ratio_ev_landscape.csv}
        \caption{
            Order of convergence of the landscape error estimator.
            Here, \(\eta \approx C N(h)^{-\operatorname{ord}}\).
        }
    \end{center}
\end{table}

% Pictures of the mesh during the adaptive algorithm, under landscape-driven refinement.
\begin{figure}[htbp]
    \begin{subfigure}[t]{0.49\textwidth}
        \centering
        \includegraphics[width=\textwidth]{mesh/mesh_landscape_0.png}
        \caption{Initial mesh.}
        \label{fig:mesh_0}
    \end{subfigure}
    \begin{subfigure}[t]{0.49\textwidth}
        \centering
        % \includegraphics[width=0.5\textwidth, trim={30cm 60cm 30cm 60cm}, clip]{mesh_10.png}
        \includegraphics[width=\textwidth]{mesh/mesh_landscape_10.png}
        \caption{Mesh after ten iterations.}
        \label{fig:mesh_10}
    \end{subfigure}
    \vspace{1cm}
    \begin{subfigure}[t]{0.49\textwidth}
        \centering
        % \includegraphics[width=0.5\textwidth, trim={30cm 60cm 30cm 60cm}, clip]{mesh_20.png}
        \includegraphics[width=\textwidth]{mesh/mesh_landscape_20.png}
        \caption{Mesh after twenty iterations.}
        \label{fig:mesh_20}
    \end{subfigure}
    \begin{subfigure}[t]{0.49\textwidth}
        \centering
        % \includegraphics[width=0.5\textwidth, trim={30cm 60cm 30cm 60cm}, clip]{mesh_30.png}
        \includegraphics[width=\textwidth]{mesh/mesh_landscape_30.png}
        \caption{Mesh after thirty iterations.}
        \label{fig:mesh_30}
    \end{subfigure}
    \vspace{1cm}
    \begin{subfigure}[t]{0.49\textwidth}
        \centering
        % \includegraphics[width=0.5\textwidth, trim={30cm 60cm 30cm 60cm}, clip]{mesh_40.png}
        \includegraphics[width=\textwidth]{mesh/mesh_landscape_40.png}
        \caption{Mesh after forty iterations.}
        \label{fig:mesh_40}
    \end{subfigure}
    \begin{subfigure}[t]{0.49\textwidth}
        \centering
        % \includegraphics[width=0.5\textwidth, trim={30cm 60cm 30cm 60cm}, clip]{mesh_49.png}
        \includegraphics[width=\textwidth]{mesh/mesh_landscape_44.png}
        \caption{Final mesh.}
        \label{fig:mesh_50}
    \end{subfigure}
    \caption{Meshes during the adaptive algorithm.}
\end{figure}


% Eigenvalues, after landscape-driven refinement. Real part.
\begin{figure}[htbp]
    \begin{subfigure}[t]{0.49\textwidth}
        \centering
        \includegraphics[width=\textwidth]{landscape/ev0}
        \caption{Eigenfunction \(\psi_0\).}
        \label{fig:eigenfunction_0}
    \end{subfigure}
    \begin{subfigure}[t]{0.49\textwidth}
        \centering
        \includegraphics[width=\textwidth]{landscape/ev1}
        \caption{Eigenfunction \(\psi_1\).}
        \label{fig:eigenfunction_1}
    \end{subfigure}
    \vspace{1cm}
    \begin{subfigure}[t]{0.49\textwidth}
        \centering
        \includegraphics[width=\textwidth]{landscape/ev2}
        \caption{Eigenfunction \(\psi_2\).}
        \label{fig:eigenfunction_2}
    \end{subfigure}
    \begin{subfigure}[t]{0.49\textwidth}
        \centering
        \includegraphics[width=\textwidth]{landscape/ev3}
        \caption{Eigenfunction \(\psi_3\).}
        \label{fig:eigenfunction_3}
    \end{subfigure}
    \vspace{1cm}
    \begin{subfigure}[t]{0.49\textwidth}
        \centering
        \includegraphics[width=\textwidth]{landscape/ev4}
        \caption{Eigenfunction \(\psi_4\).}
        \label{fig:eigenfunction_4}
    \end{subfigure}
    \begin{subfigure}[t]{0.49\textwidth}
        \centering
        \includegraphics[width=\textwidth]{landscape/ev5}
        \caption{Eigenfunction \(\psi_5\).}
        \label{fig:eigenfunction_5}
    \end{subfigure}
    \vspace{1cm}
    \begin{subfigure}[t]{0.49\textwidth}
        \centering
        \includegraphics[width=\textwidth]{landscape/ev6}
        \caption{Eigenfunction \(\psi_6\).}
        \label{fig:eigenfunction_6}
    \end{subfigure}
    \caption{(Real part of the) eigenfunctions \(\psi_i\) for \(i = 0, \ldots, 6\) at the final iteration of the adaptive algorithm.}
\end{figure}


% Eigenvalues, after landscape-driven refinement. Absolute value.
\begin{figure}[htbp]
    \begin{subfigure}[t]{0.49\textwidth}
        \centering
        \includegraphics[width=\textwidth]{landscape/abs_ev0}
        \caption{Eigenfunction \(\psi_0\).}
        \label{fig:eigenfunction_0}
    \end{subfigure}
    \begin{subfigure}[t]{0.49\textwidth}
        \centering
        \includegraphics[width=\textwidth]{landscape/abs_ev1}
        \caption{Eigenfunction \(\psi_1\).}
        \label{fig:eigenfunction_1}
    \end{subfigure}
    \vspace{1cm}
    \begin{subfigure}[t]{0.49\textwidth}
        \centering
        \includegraphics[width=\textwidth]{landscape/abs_ev2}
        \caption{Eigenfunction \(\psi_2\).}
        \label{fig:eigenfunction_2}
    \end{subfigure}
    \begin{subfigure}[t]{0.49\textwidth}
        \centering
        \includegraphics[width=\textwidth]{landscape/abs_ev3}
        \caption{Eigenfunction \(\psi_3\).}
        \label{fig:eigenfunction_3}
    \end{subfigure}
    \vspace{1cm}
    \begin{subfigure}[t]{0.49\textwidth}
        \centering
        \includegraphics[width=\textwidth]{landscape/abs_ev4}
        \caption{Eigenfunction \(\psi_4\).}
        \label{fig:eigenfunction_4}
    \end{subfigure}
    \begin{subfigure}[t]{0.49\textwidth}
        \centering
        \includegraphics[width=\textwidth]{landscape/abs_ev5}
        \caption{Eigenfunction \(\psi_5\).}
        \label{fig:eigenfunction_5}
    \end{subfigure}
    \vspace{1cm}
    \begin{subfigure}[t]{0.49\textwidth}
        \centering
        \includegraphics[width=\textwidth]{landscape/abs_ev6}
        \caption{Eigenfunction \(\psi_6\).}
        \label{fig:eigenfunction_6}
    \end{subfigure}
    \caption{Norm of the eigenfunctions \(\psi_i\) for \(i = 0, \ldots, 6\) at the final iteration of the adaptive algorithm.}
\end{figure}


%%% eigenvalue-driven refinement

% (Real part of) eigenvalues of the Laplacian on the unit square with Dirichlet boundary conditions
% close to the interval \([50\pi^2 - 10, 50\pi^2 + 10]\).
\begin{figure}[H]
    \centering
    \begin{tikzpicture}
        \begin{semilogxaxis}[
            xlabel={Number of degrees of freedom},
            ylabel={Eigenvalues},
            forget plot style={black!75, dotted},
            legend pos=south west,
            grid=major,
            cycle list name=cblist,
            xmin=150,
            xmax=120000,
            clip=false,
            legend entries={
                {\(\lambda_{i,n}\)},
                {\(\lambda_{i,N}\)},
            },
        ]
        % Add data
        \addplot+ [only marks] table[x=Ndofs, y=Evals, col sep=comma] {eigenvalues_eigenvalues.csv};

        % Add horizontal lines
        \addplot[forget plot, mark=none, samples=20] coordinates { (159, 485.7222447031804)  (120000, 485.7222447031804)};
        \addplot[forget plot, mark=none, samples=20] coordinates { (159, 490.1641386920197)  (120000, 490.1641386920197)};
        \addplot[forget plot, mark=none, samples=20] coordinates { (159, 493.48390547991005) (120000, 493.48390547991005)};
        \addplot[forget plot, mark=none, samples=20] coordinates { (159, 493.48401556963995) (120000, 493.48401556963995)};
        \addplot[forget plot, mark=none, samples=20] coordinates { (159, 493.48452662344147) (120000, 493.48452662344147)};
        \addplot[forget plot, mark=none, samples=20] coordinates { (159, 499.24505256207357) (120000, 499.24505256207357)};
        \addplot[forget plot, mark=none, samples=20] coordinates { (159, 502.3049699107303)  (120000, 502.3049699107303)};

        % Add markers
        \node [pin=right:{\(\lambda_{0}\)}] at (axis cs: 100000, 485.7222447031804) {};
        \node [pin=right:{\(\lambda_{1}\)}] at (axis cs: 100000, 490.1641386920197) {};
        \node [pin=right:{\(\lambda_{2, 3, 4}\)}] at (axis cs: 100000, 493.48390547991005) {};
        \node [pin=right:{\(\lambda_{5}\)}] at (axis cs: 100000, 499.24505256207357) {};
        \node [pin=right:{\(\lambda_{6}\)}] at (axis cs: 100000, 502.3049699107303) {};

        \end{semilogxaxis}
    \end{tikzpicture}
    \caption{
        Eigenvalues inside the interval \([50\pi^2 - 10, 50\pi^2 + 10]\).
        \(\lambda_{i,n}\) denotes the \(i\)-th eigenvalue at \(n\)-th iteration, and \(\lambda_{i,N}\) denotes the \(i\)-th eigenvalue at the final iteration.
        Here, \(\lambda_{0,N} = 485.7222447031804\), \(\lambda_{1,N} = 490.1641386920197\), \(\lambda_{2,N} \approx \lambda_{3,N} \approx \lambda_{4,N} \approx 50\pi^2\), \(\lambda_{5,N} = 499.24505256207357\), and \(\lambda_{6,N} = 502.3049699107303\).
    }
    \label{fig:eigenvalues}
\end{figure}


% Evolution of the error estimator \(\eta := \eta(u)\) as a function of the number of degrees of freedom.
\begin{figure}[H]
    \centering
    \begin{tikzpicture}
        \begin{loglogaxis}[
            xlabel={Number of degrees of freedom},
            ylabel={Error estimator},
            forget plot style={black!75, dotted},
            legend entries={\(\eta\)},
            legend pos=south west,
            cycle list name=cblist,
            grid=major,
        ]
        \addplot+ table[x=Ndofs, y={Eta}, col sep=comma] {error_estimator_eigenvalues.csv};
        \addplot [forget plot] table [
            col sep=comma,
            x=Ndofs,
            y={create col/linear regression={
                y=Eta
                % variance list={1000,800,600,500,400,200,100}
                }
            }] {error_estimator_eigenvalues.csv}
            coordinate [pos=0.8] (A)
            coordinate [pos=0.9] (B);
        \xdef\slope{\pgfplotstableregressiona}
        \draw (A) -| (B) node[pos=0.75, anchor=west] {\(\pgfmathprintnumber{\slope}\)};
        \end{loglogaxis}
    \end{tikzpicture}
    \caption{Error estimator \(\eta := \eta(u)\) as a function of the number of degrees of freedom.}
    \label{fig:error}
\end{figure}

% Monitoring the behavior of the error estimator \(\eta_i := \eta(\psi_i, \lambda_i)\) as a function of the number of degrees of freedom.
\begin{figure}[H]
    \centering
    \begin{tikzpicture}
        \begin{loglogaxis}[
            xlabel={Number of degrees of freedom},
            ylabel={Error estimator},
            legend entries={{\(\eta_0\)},
                            {\(\eta_1\)},
                            {\(\eta_2\)},
                            {\(\eta_3\)},
                            {\(\eta_4\)},
                            {\(\eta_5\)},
                            {\(\eta_6\)}},
            legend pos=south west,
            cycle list name=cblist,
            grid=major,
        ]
        % \addplot table[x=Ndofs, y={Eta}, col sep=comma] {error_estimator.csv};
        \addplot+ table[x=Ndofs, y=Eta Eigenvalue 0, col sep=comma] {error_estimator_eigenvalues.csv};
        \addplot+ table[x=Ndofs, y=Eta Eigenvalue 1, col sep=comma] {error_estimator_eigenvalues.csv};
        \addplot+ table[x=Ndofs, y=Eta Eigenvalue 2, col sep=comma] {error_estimator_eigenvalues.csv};
        \addplot+ table[x=Ndofs, y=Eta Eigenvalue 3, col sep=comma] {error_estimator_eigenvalues.csv};
        \addplot+ table[x=Ndofs, y=Eta Eigenvalue 4, col sep=comma] {error_estimator_eigenvalues.csv};
        \addplot+ table[x=Ndofs, y=Eta Eigenvalue 5, col sep=comma] {error_estimator_eigenvalues.csv};
        \addplot+ table[x=Ndofs, y=Eta Eigenvalue 6, col sep=comma] {error_estimator_eigenvalues.csv};

        \addplot [forget plot] table[
            x=Ndofs,
            y={create col/linear regression={y=Eta Eigenvalue 0}},
            col sep=comma] {error_estimator_eigenvalues.csv}
            coordinate [pos=0.8] (A)
            coordinate [pos=0.9] (B);
        \xdef\slope{\pgfplotstableregressiona}
        \draw (A) -| (B) node[pos=0.75, anchor=west] {\(\pgfmathprintnumber{\slope}\)};

        \end{loglogaxis}
    \end{tikzpicture}
    \caption{Error estimator \(\eta_i := \eta(\psi_i, \lambda_i)\) as a function of the number of degrees of freedom.}
    \label{fig:error}
\end{figure}


% Visualizing the error of the eigenvalues, after landscape-driven refinement.
\begin{figure}[H]
    \centering
    \begin{tikzpicture}
        \begin{loglogaxis}[
            xlabel={Number of degrees of freedom},
            ylabel={\(\left\lvert \lambda_{i,n} - \lambda_{i,N} \right\rvert\)},
            cycle list name=cblist,
            legend pos=south west,
            grid=major,
            % xmin=150, xmax=257000,
            legend entries={\(\operatorname{err}_0\),
                            \(\operatorname{err}_1\),
                            \(\operatorname{err}_2\),
                            \(\operatorname{err}_3\),
                            \(\operatorname{err}_4\),
                            \(\operatorname{err}_5\),
                            \(\operatorname{err}_6\)},
        ]
        \addplot+ table[x=Ndofs, y=Error Eigenvalue 0, col sep=comma] {error_eigenvalues_eigenvalues.csv};
        \addplot+ table[x=Ndofs, y=Error Eigenvalue 1, col sep=comma] {error_eigenvalues_eigenvalues.csv};
        \addplot+ table[x=Ndofs, y=Error Eigenvalue 2, col sep=comma] {error_eigenvalues_eigenvalues.csv};
        \addplot+ table[x=Ndofs, y=Error Eigenvalue 3, col sep=comma] {error_eigenvalues_eigenvalues.csv};
        \addplot+ table[x=Ndofs, y=Error Eigenvalue 4, col sep=comma] {error_eigenvalues_eigenvalues.csv};
        \addplot+ table[x=Ndofs, y=Error Eigenvalue 5, col sep=comma] {error_eigenvalues_eigenvalues.csv};
        \addplot+ table[x=Ndofs, y=Error Eigenvalue 6, col sep=comma] {error_eigenvalues_eigenvalues.csv};

        \addplot [forget plot] table[
            x=Ndofs,
            y={create col/linear regression={y=Error
            Eigenvalue 0}},
            col sep=comma] {error_eigenvalues_eigenvalues.csv}
            coordinate [pos=0.8] (A)
            coordinate [pos=0.9] (B);
        \xdef\slope{\pgfplotstableregressiona}
        \draw (A) -| (B) node[pos=0.75, anchor=west] {\(\pgfmathprintnumber{\slope}\)};
        \end{loglogaxis}
    \end{tikzpicture}
    \caption{Error \(\operatorname{err}_i := \left\lvert \lambda_{i,n} - \lambda_{i,N} \right\rvert\) as a function of the number of degrees of freedom.}
    \label{fig:error_alt}
\end{figure}


% Table of convergence ratios, under landscape-driven refinement.
% TODO: Fix scientific notation!
% \begin{table}[h!]
%     \begin{center}
%         \label{tab:convergence}
%         \pgfplotstabletypeset[
%             col sep=comma,
%             columns={Ndofs, Ratio 0, Ratio 1, Ratio 2, Ratio 3, Ratio 4, Ratio 5, Ratio 6},
%             columns/Ndofs/.style={
%                 column name={\#\textsc{DoFs}},
%                 int detect,
%                 column type={c}
%             },
%             columns/Ratio 0/.style={
%                 column name={\(\operatorname{ord}_0\)},
%                 precision=2,
%                 fixed zerofill,
%                 column type={c}
%             },
%             columns/Ratio 1/.style={
%                 column name={\(\operatorname{ord}_1\)},
%                 precision=2,
%                 fixed zerofill,
%                 column type={c}
%             },
%             columns/Ratio 2/.style={
%                 column name={\(\operatorname{ord}_2\)},
%                 precision=2,
%                 fixed zerofill,
%                 column type={c}
%             },
%             columns/Ratio 3/.style={
%                 column name={\(\operatorname{ord}_3\)},
%                 precision=2,
%                 fixed zerofill,
%                 column type={c}
%             },
%             columns/Ratio 4/.style={
%                 column name={\(\operatorname{ord}_4\)},
%                 precision=2,
%                 fixed zerofill,
%                 column type={c}
%             },
%             columns/Ratio 5/.style={
%                 column name={\(\operatorname{ord}_5\)},
%                 precision=2,
%                 fixed zerofill,
%                 column type={c}
%             },
%             columns/Ratio 6/.style={
%                 column name={\(\operatorname{ord}_6\)},
%                 precision=2,
%                 fixed zerofill,
%                 column type={c}
%             },
%             ]{ratio_ev_landscape.csv}
%             ]{ratio_ev_eigenvalues.csv}
%         \caption{
%             Order of convergence of the eigenvalues.
%             Here, \(\operatorname{err}_i \approx C_i N(h)^{-\operatorname{ord}_i}\).
%         }
%     \end{center}
% \end{table}

% Table of convergence (decay) ratio for the landscape error estimator.
\begin{table}
    \begin{center}
        \label{tab:error}
        \pgfplotstabletypeset[
            col sep=comma,
            columns={Ndofs, Ratio},
            columns/Ndofs/.style={
                column name={\#\textsc{DoFs}},
                int detect,
                column type={c}
            },
            columns/Ratio/.style={
                column name={\(\operatorname{ord}\)},
                precision=2,
                fixed zerofill,
                column type={c}
            },
            ]{ratio_ev_eigenvalues.csv}
        \caption{
            Order of convergence of the landscape error estimator.
            Here, \(\eta \approx C N(h)^{-\operatorname{ord}}\).
        }
    \end{center}
\end{table}

% Pictures of the mesh during the adaptive algorithm, under landscape-driven refinement.
\begin{figure}[htbp]
    \begin{subfigure}[t]{0.49\textwidth}
        \centering
        \includegraphics[width=\textwidth]{mesh/mesh_eigenvalues_0.png}
        \caption{Initial mesh.}
        \label{fig:mesh_0}
    \end{subfigure}
    \begin{subfigure}[t]{0.49\textwidth}
        \centering
        % \includegraphics[width=0.5\textwidth, trim={30cm 60cm 30cm 60cm}, clip]{mesh_10.png}
        \includegraphics[width=\textwidth]{mesh/mesh_eigenvalues_10.png}
        \caption{Mesh after ten iterations.}
        \label{fig:mesh_10}
    \end{subfigure}
    \vspace{1cm}
    \begin{subfigure}[t]{0.49\textwidth}
        \centering
        % \includegraphics[width=0.5\textwidth, trim={30cm 60cm 30cm 60cm}, clip]{mesh_20.png}
        \includegraphics[width=\textwidth]{mesh/mesh_eigenvalues_20.png}
        \caption{Mesh after twenty iterations.}
        \label{fig:mesh_20}
    \end{subfigure}
    \begin{subfigure}[t]{0.49\textwidth}
        \centering
        % \includegraphics[width=0.5\textwidth, trim={30cm 60cm 30cm 60cm}, clip]{mesh_30.png}
        \includegraphics[width=\textwidth]{mesh/mesh_eigenvalues_30.png}
        \caption{Mesh after thirty iterations.}
        \label{fig:mesh_30}
    \end{subfigure}
    \vspace{1cm}
    \begin{subfigure}[t]{0.49\textwidth}
        \centering
        % \includegraphics[width=0.5\textwidth, trim={30cm 60cm 30cm 60cm}, clip]{mesh_40.png}
        \includegraphics[width=\textwidth]{mesh/mesh_eigenvalues_40.png}
        \caption{Mesh after forty iterations.}
        \label{fig:mesh_40}
    \end{subfigure}
    \begin{subfigure}[t]{0.49\textwidth}
        \centering
        % \includegraphics[width=0.5\textwidth, trim={30cm 60cm 30cm 60cm}, clip]{mesh_49.png}
        \includegraphics[width=\textwidth]{mesh/mesh_eigenvalues_44.png}
        \caption{Final mesh.}
        \label{fig:mesh_50}
    \end{subfigure}
    \caption{Meshes during the adaptive algorithm.}
\end{figure}


% Eigenvalues, after landscape-driven refinement. Real part.
\begin{figure}[htbp]
    \begin{subfigure}[t]{0.49\textwidth}
        \centering
        \includegraphics[width=\textwidth]{eigenvalue/ev0}
        \caption{Eigenfunction \(\psi_0\).}
        \label{fig:eigenfunction_0}
    \end{subfigure}
    \begin{subfigure}[t]{0.49\textwidth}
        \centering
        \includegraphics[width=\textwidth]{eigenvalue/ev1}
        \caption{Eigenfunction \(\psi_1\).}
        \label{fig:eigenfunction_1}
    \end{subfigure}
    \vspace{1cm}
    \begin{subfigure}[t]{0.49\textwidth}
        \centering
        \includegraphics[width=\textwidth]{eigenvalue/ev2}
        \caption{Eigenfunction \(\psi_2\).}
        \label{fig:eigenfunction_2}
    \end{subfigure}
    \begin{subfigure}[t]{0.49\textwidth}
        \centering
        \includegraphics[width=\textwidth]{eigenvalue/ev3}
        \caption{Eigenfunction \(\psi_3\).}
        \label{fig:eigenfunction_3}
    \end{subfigure}
    \vspace{1cm}
    \begin{subfigure}[t]{0.49\textwidth}
        \centering
        \includegraphics[width=\textwidth]{eigenvalue/ev4}
        \caption{Eigenfunction \(\psi_4\).}
        \label{fig:eigenfunction_4}
    \end{subfigure}
    \begin{subfigure}[t]{0.49\textwidth}
        \centering
        \includegraphics[width=\textwidth]{eigenvalue/ev5}
        \caption{Eigenfunction \(\psi_5\).}
        \label{fig:eigenfunction_5}
    \end{subfigure}
    \vspace{1cm}
    \begin{subfigure}[t]{0.49\textwidth}
        \centering
        \includegraphics[width=\textwidth]{eigenvalue/ev6}
        \caption{Eigenfunction \(\psi_6\).}
        \label{fig:eigenfunction_6}
    \end{subfigure}
    \caption{(Real part of the) eigenfunctions \(\psi_i\) for \(i = 0, \ldots, 6\) at the final iteration of the adaptive algorithm.}
\end{figure}


% Eigenvalues, after landscape-driven refinement. Absolute value.
\begin{figure}[htbp]
    \begin{subfigure}[t]{0.49\textwidth}
        \centering
        \includegraphics[width=\textwidth]{eigenvalue/abs_ev0}
        \caption{Eigenfunction \(\psi_0\).}
        \label{fig:eigenfunction_0}
    \end{subfigure}
    \begin{subfigure}[t]{0.49\textwidth}
        \centering
        \includegraphics[width=\textwidth]{eigenvalue/abs_ev1}
        \caption{Eigenfunction \(\psi_1\).}
        \label{fig:eigenfunction_1}
    \end{subfigure}
    \vspace{1cm}
    \begin{subfigure}[t]{0.49\textwidth}
        \centering
        \includegraphics[width=\textwidth]{eigenvalue/abs_ev2}
        \caption{Eigenfunction \(\psi_2\).}
        \label{fig:eigenfunction_2}
    \end{subfigure}
    \begin{subfigure}[t]{0.49\textwidth}
        \centering
        \includegraphics[width=\textwidth]{eigenvalue/abs_ev3}
        \caption{Eigenfunction \(\psi_3\).}
        \label{fig:eigenfunction_3}
    \end{subfigure}
    \vspace{1cm}
    \begin{subfigure}[t]{0.49\textwidth}
        \centering
        \includegraphics[width=\textwidth]{eigenvalue/abs_ev4}
        \caption{Eigenfunction \(\psi_4\).}
        \label{fig:eigenfunction_4}
    \end{subfigure}
    \begin{subfigure}[t]{0.49\textwidth}
        \centering
        \includegraphics[width=\textwidth]{eigenvalue/abs_ev5}
        \caption{Eigenfunction \(\psi_5\).}
        \label{fig:eigenfunction_5}
    \end{subfigure}
    \vspace{1cm}
    \begin{subfigure}[t]{0.49\textwidth}
        \centering
        \includegraphics[width=\textwidth]{eigenvalue/abs_ev6}
        \caption{Eigenfunction \(\psi_6\).}
        \label{fig:eigenfunction_6}
    \end{subfigure}
    \caption{Norm of the eigenfunctions \(\psi_i\) for \(i = 0, \ldots, 6\) at the final iteration of the adaptive algorithm.}
\end{figure}
